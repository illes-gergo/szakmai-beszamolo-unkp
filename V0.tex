\documentclass[12pt, a4paper]{article}
\usepackage{times}
\usepackage[T1]{fontenc}
\usepackage[margin = 1in]{geometry}
\usepackage[magyar]{babel}
\usepackage{amsmath}
\usepackage{amssymb}
\usepackage{float}
\usepackage[style=ieee]{biblatex}
\usepackage{csquotes}
\usepackage{hyperref}
\addbibresource{irodalomjegyzek.bib}
\title{UNKP Szakmai beszámoló}
\author{Illés Gergő - HHU75J}
\date{}
\begin{document}

\maketitle
\section{A kutatási tervben említettek teljesítése}
\subsection{Első szakasz (5 hónap)}
A szeptembertől januárig tartó 5 hónapra terveztük a \textit{,,Terahertzes impulzusok előállítása mikrostrukturált lítium niobát kristályban''} című TDK dolgozatban bemutatott eredményeit felhasználva illetve azokat bővítve egy nemzetközi publikációt megírni. Ez meg is valósult a 2022. március 22. és 25. között az amerikai Optica (előzőleg OSA) által megrendezésre került High-Brightness Sources and Light-Driven Interactions Congress keretein belül \cite{illes2022terahertz}. A konferencián az anyag poszter formájában került bemutatásra, amely poszter kéziratának leadási határideje 2021. december 21. volt. A poszteren szerepeltek a TDK dolgozatban bemutatott eredmények amelyek bemutatták, hogy az egyes geometriai paraméterek megváltoztatása hogyan befolyásolják a keletkező terahertzes (továbbiakban THz-es) impulzusok alakját a nyaláb keresztmetszete mentén. A poszter tárgyalta még az elrendezésben használt optikai rács orientációjának fontosságát, számítási eredményeket mutat be arról, hogy a Littrow-szögtől való eltérés függvényében hogyan változik az elrendezés hatásfoka. Új eredményként bemutatásra került egy olyan számítássorozat amely vizsgálja azt, hogy a kristályhossz és a pumpa nyalábméretének függvényében hogy alakul egy optimalizált elrendezés hatásfoka, illetve az elrendezés által keltett THz-es impulzus energiája.\\
Ezzel egy időben megkezdtük egy új THz-es numerikus modell elkészítését amely már figyelembe veszi a THz-es jel visszahatását a pumpaimpulzusra. Ezen modell elkészítéshez \cite{ravi2014limitations} cikket vettük alapul. A modell elkészítésekor a nehézséget a csatolt differenciálegyenlet rendszer megoldása okozta. Ehhez saját magam írtam egy olyan kódot ami tetszőleges számú tagból álló első fokú differenciálegyenlet-rendszert képes megoldani negyed rendű Runge-Kutta módszerrel. Megkezdtük a kód tesztelését technikai szempontból. Vizsgáltuk a szükséges térbeli, illetve időbeli felbontást olyan tekintetben, hogy milyen felbontások szükségesek a stabil megoldásához. Fizikailag és informatikailag optimalizáltuk a kódot gyorsabb futási idők elérése érdekében. Egyes számítások amelyek ezen kód készültségének korai fázisában készültek szintén bemutatásra kerültek a igh-Brightness Sources and Light-Driven Interactions Congress-en rövid előadás formájában \cite{nasi2022comparison}.
\section{Második szakasz (5 hónap)}
A második szakaszban ezen 1D+1-es kód továbbfejlesztésén dolgoztam elsősorban, valamint részletesen vizsgáltam a kapott eredményeket az egyszerűbb visszahatás nélküli modellel összehasonlítva. Ezen összehasonlítások elsősorban a keltési hatásfokra, valamint a THz-es térerősség időbeli alakjára vonatkoztak. Ezek a számítások  azt mutatták hogy a döntött impulzusfrontú gerjesztés elrendezésnél a visszahatás következtében a hatásfok átlagosan ötödére csökkent, a kristályhossz amelynél a hatásfok maximális szintén átlagosan a negyedére esett vissza. A \textit{,,Terahertzes impulzusok terjedésének és keltésének modellezése''} című szakdolgozatomban részletesen beszámoltam a két  modell közti különbségről.\\
Tovább vizsgáltam az új visszahatásos modellt egy optimális beállítást keresve. Optimalizáltam az elrendezést a pumpaimpulzus hossza, pumpáló intenzitás és az impulzusfront döntés szöge szerint. Megvizsgáltam hogy ezen paraméterek hogyan alakítják a keltési hatásfokot, valamint vizsgáltam a THz-es impulzus időbeli alakját és spektrumát.\\
A szakdolgozat írásával egy időben elkezdtünk fejleszteni egy olyan modellt ami már 2 térbeli dimenzióban számol. Ezen modellt alapját \cite{ravi2015theory} képzi. Az itt bemutatott módszer előnyös, mert a parciális differenciálegyenleteket Fourier-transzformálásokon keresztül közönséges differenciálegyenletekké alakíthatók. Így nincs szükség végeselem módszerek használatára ami azt jelenti hogy a számítások gyorsabban futhatnak kevesebb memóriahasználat mellett. A cikkben azonban vannak matematikai hibák, amelyek lassították a modell fejlesztését. Nehézséget okoz továbbá az is, hogy a két dimenzió miatt a számítások sok memóriát használnak fel annak ellenére hogy nem végeselem módszerekkel dolgozunk. Ezt úgy tudjuk enyhíteni ha csökkentjük a térbeli és időbeli felbontást, ekkor azonban már jelentkezhetnek numerikus hibák. A modell eddig a döntött frontú pumpaimpulzus lineáris terjedését képes modellezni, aminek helyességét analitikus formulák által kapott eredményekkel validáltuk.\\
A modellt további munka során be fogjuk fejezni, mert számos lehetőséget tartogat. Számunkra a legfontosabb az, hogy ezzel a modellel már képesek leszünk olyan elrendezéseket modellezni nagy pontossággal amelyek felülete mikroszkopikusan megmunkált mint például a leképzés nélküli echelon elrendezés \cite{toth2019numerical}.
\printbibliography[heading = bibintoc, title = Hivatkozások]
\end{document}